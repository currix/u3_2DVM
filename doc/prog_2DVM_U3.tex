% Created 2019-05-03 vie 17:28
\documentclass[a4paper,12pt,captions=tableheading]{article}
\usepackage[utf8]{inputenc}
\usepackage[T1]{fontenc}
\usepackage{fixltx2e}
\usepackage{graphicx}
\usepackage{longtable}
\usepackage{float}
\usepackage{wrapfig}
\usepackage{rotating}
\usepackage[normalem]{ulem}
\usepackage{amsmath}
\usepackage{textcomp}
\usepackage{marvosym}
\usepackage{wasysym}
\usepackage{amssymb}
\usepackage{hyperref}
\tolerance=1000
\usepackage{authblk}
\author[1]{Curro Perez-Bernal\thanks{curropb@uhu.es}}
\affil[1]{Depto.\ CC.\ Integradas y Centro de Estudios Avanzados en F\'isica, Matem\'aticas y Computaci\'on, Fac.\ CC. Experimentales, Universidad de Huelva}
\author[1]{Jamil Khalouf Rivera}
\author[1]{Miguel Carvajal Zaera}
\usepackage[T1]{fontenc}
\usepackage[utf8]{inputenc}
\usepackage{siunitx}
\usepackage{mathptmx}
\usepackage{pifont}
\usepackage[margin=0.7in]{geometry}
\usepackage[backend=bibtex,sorting=none]{biblatex}
\bibliography{rdiary.bib}
\newcommand{\point}[1]{\noindent \textbf{#1}}
\usepackage{hyperref}
\usepackage{csquotes}
\usepackage[mla]{ellipsis}
\parindent = 0em
\setlength\parskip{.5\baselineskip}
\usepackage{physics}
\date{\today}
\title{2D Vibron Model Program Suite}
\hypersetup{
  pdfkeywords={},
  pdfsubject={},
  pdfcreator={}}
\begin{document}

\maketitle
\tableofcontents


\section{Abstract}
\label{sec-1}

This is the documentation of the \verb~2DVM_U3~ program suite that is composed of mainly \verb~Fortran 90~ but also \verb~Perl~ and \verb~Python~ scripts to 
perform calculations using the 2D limit of the Vibron model (2DVM).


\section{Theory Outline}
\label{sec-2}

The original paper where the 2DVM was presented is Ref.\textasciitilde{}\cite{Iachello1996}. We
also make extensive use of the detailed description of the model
provided in Ref.\textasciitilde{}\cite{PBernal2008}. Other references deal mainly with
the application of the 2DVM to molecular bending vibrations
\cite{Ishikawa2002,Iachello2003,PBernal2005}, quantum phase transitions
\cite{Caprio2008,PBernal2010,PFernandez2011,PRA86,Santos2015,santos2016,Castanos2016,PB_Santos2016},
or both \cite{Larese2011,algmonod1}. New developments also consider the coupling
of several \(U(3)\) dynamical algebras for the modeling of systems
with several degrees of freedom
\cite{mp_u3xu3,pla_376,cocoyoc,Larese2014,Calixto2014}.

We proceed to review the main theory results used in this program suite. 

The 2DVM has an \(U(3)\) dynamical algebra whose nine generators are
built as bilinear products of one scalar \(\sigma^\dagger\) and two
circular \(\tau_\pm^\dagger\) bosonic creation operators with the corresponding
annihilation operators, \(\sigma,\tau_\pm\) \cite{PBernal2008}. The nine generators are
recasted to provide a deeper physical insight as \cite{Iachello1996}

\begin{equation}
  \begin{array}{lcl}
    \hat n = \tau^\dagger_+\tau_++\tau^\dagger_-\tau_- &  , &
    \hat n_s = \sigma^\dagger\sigma 
    \\
    \hat \ell = \tau^\dagger_+\tau_+-\tau^\dagger_-\tau_- &, &
    \hat Q_\pm = \sqrt{2}\,\tau^\dagger_\pm\tau_\mp \\
    \hat D_\pm = \sqrt{2}(\pm\tau^\dagger_\pm\sigma\mp\sigma^\dagger\tau_\mp) &, &
    \hat R_\pm =\sqrt{2} (\tau^\dagger_\pm\sigma+\sigma^\dagger\tau_\mp) ~~.
  \end{array}
  \label{gen}
\end{equation}

There are two possible subalgebra chains starting from the dynamical
algebra \(U(3)\) and ending in the symmetry algebra \(SO(2)\). The
symmetry algebra has the 2D angular momentum \(\hat \ell\) as
generator, which imply that this is a conserved quantity:
\begin{equation}
\begin{array}{cccccl}
     &         &U(2) &          & &Chain (I)\\
     & \nearrow&     & \searrow & &\\
U(3) &         &     &          & SO(2)&\\
     & \searrow&     & \nearrow & &\\
     &         &SO(3)&          & &Chain(II)\\
\end{array}
\end{equation}
The corresponding  subalgebras are composed by the following elements
\begin{equation}
\begin{array}{cc}
U(2)  &\{\hat n, \hat \ell, \hat Q_+, \hat Q_-\}\\
SO(3) &\{\hat l, \hat D_+, \hat D_-\}\\ 
SO(2) &\{\hat l\}
\end{array}
\label{chaingen}
\end{equation}
\noindent where the $SO(3)$ elements satisfy the usual angular momentum
commutation relations. Because of an automorphism of the Lie algebra
$U(3)$ constructed with $\tau_\pm$, $\sigma$ there is an alternative
$SO(3)$ subalgebra of $U(3)$, called $\overline{SO(3)}$, with elements

\begin{equation}
\begin{array}{cr}
\overline{SO(3)} &\{\hat l, \hat R_+, \hat R_-\} ~~.
\end{array}
\label{so3bar}
\end{equation}

Another ingredient of the algebraic approach are the Casimir (or
invariant) operators associated to each subalgebra chain
\cite{frank, bookalg}. The first and second order Casimir operators for
the subalgebras in Eqs.$\backslash$ (\ref{chaingen}) and (\ref{so3bar}) are 
\begin{equation}
\begin{array}{ccl}
\hat C_1[U(2)]=\hat n&,&\hat C_2[U(2)]= \hat n(\hat n+1)\\
\multicolumn{3}{c}{\hat C_2[SO(3)] = \hat W^2 =(\hat D_+\hat D_-+\hat D_-\hat
D_+)/2+\hat l^2} \\ 
\hat C_1[SO(2)]=\hat l &,&\hat C_2[SO(2)]=\hat l^2\\
\end{array}
\label{casimir}
\end{equation}
\noindent and 
\begin{equation}
\hat C_2[\overline{SO(3)}] = \hat {\overline{W}}^2 = (\hat R_+\hat R_-+\hat R_-\hat
R_+)/2+\hat \ell^2~~.
\label{casimirbar}
\end{equation}
We proceed to state the basis quantum numbers and branching rules for each  dynamical symmetry.

\subsection{\(U(2)\) or Cylindrical Oscillator Dynamical Symmetry}
\label{sec-2-1}

In this chain states are labeled by the quantum numbers \(n\) and \(\ell\)

\begin{equation}
\left|\begin{array}{ccccc}
U(3)&\supset& U(2)&\supset& SO(2)\\
\left[N\right]   &       & n   &       & \ell
\end{array}\right\rangle ~~,
\label{cobas}
\end{equation}
\noindent and the basis is \(|[N];n,\ell\rangle\). An alternative notation,
often used in molecular physics, is \(|[N];n^\ell\rangle\). The quantum number $N$ 
labels the totally symmetric representation of $U(3)$, $[N]$, and it is
related to the total number of bound states of the system, $n$ is the
vibrational quantum number, and $\ell$ is the vibrational angular momentum. The
branching rules are 
\begin{align}
n & =  N, N-1, N-2, \ldots, 0 \nonumber\\
l & =  \pm n, \pm (n-2), \ldots, \pm 1 \mbox{ or }0~,~~ (n = \mbox{odd or
  even}) ~~.
\end{align}

This is the natural basis to fit experimental bending vibration data
in the case of linear and quasilinear molecules.

\subsection{\(SO(3)\) or Displaced Oscillator Dynamical Symmetry}
\label{sec-2-2}

States in this chain are
characterized by the quantum numbers 
\begin{equation}
\left|\begin{array}{ccccc}
U(3)&\supset& SO(3)&\supset& SO(2)\\
\left[N\right]   &       & \omega   &       & \ell
\end{array}\right\rangle~~.
\label{anosbas}
\end{equation}
and will be labelled as $|[N];\omega,\ell\rangle$. 
The branching rules are
\begin{align}
\omega & =  N, N-2, N-4, \ldots, 1 \mbox{ or }0 ~,~~ (N = \mbox{odd or
  even}),\nonumber\\
l & =  \pm \omega, \pm (\omega-1) , \ldots , 0~~.
\end{align} 
In this case, it is
convenient to introduce a vibrational 
quantum number \(v\), which can be identified with the number of quanta of
excitation in the displaced oscillator:
\begin{equation}
v=\frac{N-\omega}{2}~~.
\label{vdef}
\end{equation}
The branching rules in this case are
\begin{eqnarray}
v & = & 0,1, \ldots, \frac{N-1}{2} \mbox{ or }\frac{N}{2}~,~~ (N = \mbox{odd or
  even}),\nonumber\\
l & = & 0, \pm 1,\pm 2, \ldots, \pm (N-2v)~~.
\end{eqnarray} 
This is the natural basis to fit experimental bending vibration data
in the case of quasi-rigid and rigidly-bent molecules.

\subsection{2DVM Hamiltonians included in the program suite}
\label{sec-2-3}
We present the three different Hamiltonians that are considered in the
present programs. The simplest Hamiltonian, used mainly in the study
of QPTs and ESQPTs, is the so called model Hamiltonian, that includes
the \(\hat n = \tau^\dagger_+\tau_++\tau^\dagger_-\tau_-\) operator
from the dynamical symmetry (I) (DS-I) and the) and the Pairing
operator \(\hat P\) which is built adding a diagonal contribution to
the \(SO(3)\) second order Casimir operator \(\hat W^2 =(\hat D_+\hat
D_-+\hat D_-\hat D_+)/2+\hat \ell^2\). The Pairing operator is \(\hat
P=N(N+1)-\hat W^2\).  the $SO(3)$ algebra).

The model Hamiltonian \(\hat {\cal H}\) is then
\begin{equation}
\hat {\cal H} = \varepsilon\left[(1-\xi)\hat n + \frac{\xi}{N-1}\hat P\right]~~.
\label{modham1}
\end{equation}
The Hamiltonian has two parameters: a global energy scale
\(\varepsilon\) and a control parameter \(\xi\) such that for \(\xi =
0.0\) the system is in the DS-I and for \(\xi = 1.0\) the system is in
the DS-I. This is specially adequate to characterize the ground state
and excited state QPTs as there is a second order ground state QPT for
the critical value \(\xi_c = 0.2\) \cite{PBernal2008}.

A second Hamiltonian is \(\hat H_{2b}\), the most general one- and two-body Hamiltonian of the 2DVM defined in \cite{Iachello1996}. 

\begin{equation}
\hat H_{2b} = E_0+\epsilon {\hat n} + \alpha {\hat n} ({\hat n}+1) + \beta
{\hat \ell}^2 + A \hat P~~,
\label{ham2b}
\end{equation}
\noindent where the operators \({\hat n}\) and \({\hat n} ({\hat
n}+1)\) belong to the DS-I and the pairing operator \(\hat P\) to the
DS-II while the vibrational angular momentum, \(\hat\ell\), is common
to both dynamical chains. In fact in all the cases considered the
angular momentum is a constant of the motion, \(\ell\) is a good
quantum number, and the Hamiltonian are block diagonal in \(\ell\), a
fact that simplifies the numerical calculations reducing the dimension
of the considered matrices. In addition to this, as in the absence of
external fields, the first order angular momentum operator
\(\hat\ell\) is not included there is a degeneracy between positive
and negative \(\ell\) values for \(\ell\ne 0\) and only \(\ell\ge0\)
are considered. The formula for the dimension of a subspace of the
totally symmetric representation \(N\) with angular momentum \(\ell\)
is \((N-|\ll|)/2 + 1\), if \(N\) and \(\ell\) have the same parity,
and \((N-|\ell|+1)/2\) if they have different parity. Both cases are
considered simultaneously in the programs as
\begin{equation}
dim(N,\ell) = \frac{N-|\ell|+mod(N-|\ell|,2)+1}{2}~~.
\end{equation}

The third and last Hamiltonian considered is \(\hat H_{4b}\), the most
general 1-, 2-, 3-, and 4-body Hamiltonian, that can be expressed in
terms of the Casimir operators \({\hat n}\), \({\hat n}^2\), \({\hat
\ell}^2\), \(\hat W^2\), and \(\hat{\overline{W}^2}\) can be expressed
as follows

\begin{align}
\hat H_{4b} =& P_{11} \hat n \nonumber\\
& + P_{21} \hat n^2 + P_{22} \hat \ell^2 + P_{23} \hat W^2 \nonumber\\
& + P_{31} \hat n^3 + P_{32} \hat n \hat \ell^2 + P_{33} (\hat n \hat W^2 + \hat W^2 \hat n) \label{H4b}\\
& + P_{41} \hat n^4 + P_{42} \hat n^2 \hat \ell^2 + P_{43} \hat \ell^4 + P_{44} \hat \ell^2 \hat W^2 \nonumber\\
& + P_{45} (\hat n^2 \hat W^2 + \hat W^2 \hat n^2) + P_{46} \hat W^4 + P_{47} (\hat {W}^2 \hat {\overline{W}}^2 + \hat {\overline{W}}^2 \hat W^2)/2~.\nonumber
\end{align}

The Hamiltonian has fourteen spectroscopic constants \(P_{ij}\), where
the subindeces indicate that this parameter corresponds to the
\(j\)-th operator of the \(i\)-body operators. 

From the matrix elements of the creation and annihilation sigma and
tau bosons in the two dynamical symmetries published in
\cite{PBernal2008}, the ensuing operator matrix elements can be
derived.


\subsubsection{Operator Matrix Elements in the Dynamical Symmetry (I)}
\label{sec-2-3-1}

The diagonal operators in this dynamical symmetry are

\begin{description}
\item[{Operator \(\hat n^p\)}] \(\langle [N];n^\ell|\hat n^p|[N];n^\ell\rangle = n^p\) for \(p = 1,2,3,4\).

\item[{Operator \(\hat \ell^{2q}\)}] \(\langle [N];n^\ell|\hat \ell^{2q}|[N];n^\ell\rangle = \ell^{2q}\) for \(q = 1,2\).

\item[{Operator \(\hat n^p\hat \ell^{2q}\)}] \(\langle [N];n^\ell|\hat n^p\ell^{2q}|[N];n^\ell\rangle = n^p\ell^{2q}\) for \(p = 1,2\) and \(q = 1\).
\end{description}

The non-diagonal matrix elements in this basis are 

\begin{description}
\item[{\(SO(3)\) Casimir Operator \(\hat W^2\)}] 
\end{description}

\begin{align}
\langle [N];n_2^l|{\hat W}^2|[N];n_1^l\rangle =&
\left[(N-n_1)(n_1+2)+(N-n_1+1)n_1 + l^2\right] \delta_{n_2,n_1} \nonumber\\
-& \sqrt{(N-n_1+2)(N-n_1+1)(n_1+l)(n_1-l)}\,\delta_{n_2,n_1-2}\label{W2matel_u2}\\
-&
\sqrt{(N-n_1)(N-n_1-1)(n_1+l+2)(n_1-l+2)}\,\delta_{n_2,n_1+2}~~.\nonumber
\end{align}

Note that this is the main nondiagonal operator in this case and it is
a band matrix as the non-zero matrix elements are located in the main and first diagonals only.

\begin{description}
\item[{Operator \(\hat n \hat W^2 + \hat W^2 \hat n\)}] As the operator
\(\hat n\) is diagonal the matrix is also band diagonal with matrix
elements
\end{description}

\begin{align}
\langle [N];n_2^l|\hat n \hat W^2 + \hat W^2 \hat n|[N];n_1^l\rangle =&
2n_1\left[(N-n_1)(n_1+2)+(N-n_1+1)n_1 + l^2\right] \delta_{n_2,n_1} \nonumber\\
-& (2n_1-2)\sqrt{(N-n_1+2)(N-n_1+1)(n_1+l)(n_1-l)}\,\delta_{n_2,n_1-2}\label{W2nmatel_u2}\\
-& (2n_1+2)\sqrt{(N-n_1)(N-n_1-1)(n_1+l+2)(n_1-l+2)}\,\delta_{n_2,n_1+2}~~.\nonumber
\end{align}

\begin{description}
\item[{Operator \(\hat \ell^2 \hat W^2\)}] This operator is computed for \(\ell\ne 0\) multiplying the \({\hat W}^2\) operator matrix times \(\ell^2\).

\item[{Operator \(\hat n^2 \hat W^2 + \hat W^2 \hat n^2\)}] This is
computed as the \(\hat n \hat W^2 + \hat W^2 \hat n\) operator.
\end{description}

\begin{align}
\langle [N];n_2^l|\hat n^2 \hat W^2 + \hat W^2 \hat n^2|[N];n_1^l\rangle =&
2n_1^2\left[(N-n_1)(n_1+2)+(N-n_1+1)n_1 + l^2\right] \delta_{n_2,n_1} \nonumber\\
-& [n_1^2 + (n_1-2)^2]\sqrt{(N-n_1+2)(N-n_1+1)(n_1+l)(n_1-l)}\,\delta_{n_2,n_1-2}\label{W2n2matel_u2}\\
-& [n_1^2 + (n_1+2)^2]\sqrt{(N-n_1)(N-n_1-1)(n_1+l+2)(n_1-l+2)}\,\delta_{n_2,n_1+2}~~.\nonumber
\end{align}

\begin{description}
\item[{Operator \(\hat W^4\)}] This operator is computed as the matrix
product of the \(\hat W^2\) operator matrix times itself.

\item[{Operator \(\hat W^2 \hat{\overline{W}}^2 + \hat{\overline{W}}^2\hat W^2 \)}] In
this basis the only difference between the matrix elements of the
\(\hat W^2\) and \(\hat{\overline{W}}^2\) operators is the sign
of the non-diagonal contribution, which is positive in this
case. The full operator is computed via matrix multiplication.
\end{description}

\subsubsection{Operator Matrix Elements in the Dynamical Symmetry (II)}
\label{sec-2-3-2}


The diagonal operators in this dynamical symmetry are

\begin{description}
\item[{\(SO(3)\) Casimir Operator \(\hat W^2\)}] \(\langle [N];\omega \ell|\hat W^2|[N]; \omega \ell\rangle = \omega(\omega + 1)\).

\item[{Operator \(\hat \ell^{2q}\)}] \(\langle [N];\omega \ell|\hat \ell^{2q}|[N];\omega \ell\rangle = \ell^{2q}\) for \(q = 1,2\).

\item[{Operator \(\hat \ell^2 \hat W^2\)}] \(\langle [N];\omega \ell|\hat \ell^2 \hat W^2|[N]; \omega \ell\rangle = \ell^2 \omega(\omega + 1\).

\item[{Operator \(\hat W^4\)}] \(\langle [N];\omega \ell|\hat W^2|[N]; \omega \ell\rangle = \omega^2(\omega + 1)^2\).
\end{description}

The non-diagonal matrix elements in this basis are 
\begin{description}
\item[{Operator \(\hat n\)}] 
\end{description}

\begin{align}
  \langle [N];w_2^l|\hat n|[N];w_1^l\rangle =&
  \left\{\frac{(N-w_1)\left[(w_1-l+2)(w_1-l+1) +
        (w_1+l+2)(w_1+l+1)\right]}{2(2w_1+1)(2w_1+3)} \right. \nonumber\\
  & + \left.\frac{(N+w_1+1)\left[(w_1+l)(w_1+l-1) +
        (w_1-l)(w_1-l-1)\right]}{2(2w_1+1)(2w_1-1)} \right\}\,
  \delta_{w_2,w_1} \label{nmatel_so3}\\
+& \sqrt{\frac{(N-w_1)(N+w_1+3)(w_1-l+2)(w_1-l+1)(w_1+l+2)(w_1+l+1)}{(2w_1+1)(2w_1+3)^2(2w_1+5)}}\delta_{w_2,w_1+2}\nonumber\\
+& \sqrt{\frac{(N-w_1+2)(N+w_1+1)(w_1-l)(w_1-l-1)(w_1+l)(w_1+l-1)}{(2w_1-3)(2w_1-1)^2(2w_1+1)}}\delta_{w_2,w_1-2}\nonumber\\
\end{align}

Note that this is the main nondiagonal operator in this case and it is again
a band matrix with non-zero matrix elements are located in the main and first diagonals only. The \(\hat n\) matrix element in this basis are taken from
\cite{PBernal2008} with a typo that has been corrected.
\begin{description}
\item[{Operators \(\hat n^2\), \(\hat n^3\), and \(\hat n^4\)}] These three operators are computed by matrix multiplication of the basic operator (\ref{nmatel_so3})

\item[{Operator \(\hat n \hat \ell^{2}\)}] This operator is computed for \(\ell\ne 0\) multiplying the \({\hat n}\) operator matrix times \(\ell^2\).

\item[{Operator \(\hat n \hat W^2 + \hat W^2 \hat n\)}] As the operator
\(\hat n\) is band diagonal \(\hat W^2\) is diagonal this operator matrix is also band diagonal with matrix
elements
\end{description}

\begin{align}
\langle [N];w_2 \ell|\hat n \hat W^2 + \hat W^2 \hat n|[N];w_1 \ell\rangle &=
 2\omega_1(\omega_1+1)  \left\{\frac{(N-w_1)\left[(w_1-l+2)(w_1-l+1) +
        (w_1+l+2)(w_1+l+1)\right]}{2(2w_1+1)(2w_1+3)} \right. \nonumber\\
  & + \left.\frac{(N+w_1+1)\left[(w_1+l)(w_1+l-1) +
        (w_1-l)(w_1-l-1)\right]}{2(2w_1+1)(2w_1-1)} \right\}\,
  \delta_{w_2,w_1} \label{nW2matel_so3}\\
+& \left[\omega_1(\omega_1+1) + (\omega_1 +2)(\omega_1+3)\right]\nonumber\\
\times &  \sqrt{\frac{(N-w_1)(N+w_1+3)(w_1-l+2)(w_1-l+1)(w_1+l+2)(w_1+l+1)}{(2w_1+1)(2w_1+3)^2(2w_1+5)}}\delta_{w_2,w_1+2}\nonumber\\
+& \left[(\omega_1 - 2)(\omega_1-1) + \omega_1 (\omega_1+1)\right]\nonumber\\
\times & \sqrt{\frac{(N-w_1+2)(N+w_1+1)(w_1-l)(w_1-l-1)(w_1+l)(w_1+l-1)}{(2w_1-3)(2w_1-1)^2(2w_1+1)}}\delta_{w_2,w_1-2}\nonumber
\end{align}


\begin{description}
\item[{Operator \(\hat n^2 \hat W^2 + \hat W^2 \hat n^2\)}] This is
computed in the same way that the \(\hat n \hat W^2 + \hat W^2
     \hat n\) operator but taking into account that the \(\hat n^2\)
operator is double banded. Therefore the operator matrix elements
can be expressed as follow
\end{description}

\begin{align}
\langle [N];w_2 \ell|\hat n^2 \hat W^2 + \hat W^2 \hat n^2|[N];w_1 \ell\rangle &=
 2\omega_1(\omega_1+1) [\hat n^2]_{w_1,w_1}\, \delta_{w_2,w_1} \nonumber\\
+& \left[\omega_1(\omega_1+1) + (\omega_1 +2)(\omega_1+3)\right][\hat n^2]_{w_1,w_1+2}\delta_{w_2,w_1+2}\nonumber\\
+& \left[(\omega_1 - 2)(\omega_1-1) + \omega_1 (\omega_1+1)\right][\hat n^2]_{w_1,w_1-2}\delta_{w_2,w_1-2}\label{n2W2matel_so3}\\
+& \left[\omega_1(\omega_1+1) + (\omega_1 +4)(\omega_1+5)\right][\hat n^2]_{w_1,w_1+4}\delta_{w_2,w_1+4}\nonumber\\
+& \left[(\omega_1 - 4)(\omega_1-3) + \omega_1 (\omega_1+1)\right][\hat n^2]_{w_1,w_1-4}\delta_{w_2,w_1-4}\nonumber~,
\end{align}
\noindent where \([\hat n^2]_{w_i,w_j}\) are the \(\hat n^2\) operator matrix elements.

\begin{description}
\item[{Operator \(\hat W^2 \hat{\overline{W}}^2 + \hat{\overline{W}}^2\hat W^2 \)}] In
this basis we need first to compute the matrix elements of the \(\hat{\overline{W}}^2\) making use of Eqs.$\backslash$ (37) and (38) of Ref.$\backslash$ \cite{PBernal2008}.
\end{description}
\begin{equation}
\langle [N];w_2 \ell_2|\hat R_{-}|[N];w_1 \ell_1\rangle = A_{w_1,\ell_1}  \delta_{w_2,w_1} \delta_{\ell_2,\ell_1-1} + B_{w_1,\ell_1}  \delta_{w_2,w_1-2} \delta_{\ell_2,\ell_1-1} + C_{w_1,\ell_1}  \delta_{w_2,w_1+2} \delta_{\ell_2,\ell_1-1}~,
\end{equation}
\noindent where 
\begin{align}
A_{w,\ell} = & \frac{(2N+3)(2\ell + 1)}{(2w-1)(2w+3)} \sqrt{(w+\ell)(w-\ell+1)/2}\nonumber\\ 
B_{w,\ell} = & -\sqrt{\frac{2(N+w+1)(N-w+2)(w+\ell)(w-\ell)(w+\ell-1)(w+\ell-2)}{(2w+1)(2w-1)^2(2w-3)}}\nonumber\\ 
C_{w,\ell} = & \sqrt{\frac{2(N+w+3)(N-w)(w+\ell+1)(w-\ell+1)(w-\ell+2)(w-\ell+3)}{(2w+1)(2w+3)^2(2w+5)}}\nonumber~.
\end{align}

The previous result can be used for the obtention of an expression for the \(\hat R_{+}\) operator matrix elements
\begin{align}
\langle [N];w_2 \ell_2|\hat R_{-}|[N];w_1 \ell_1\rangle^\dagger =& \langle [N];w_1 \ell_1|\hat R_{+}|[N];w_2 \ell_2\rangle \\
=& A_{w_2,\ell_2+1}  \delta_{w_1,w_2} \delta_{\ell_1,\ell_2+1} + B_{w_2+2,\ell_2+1}  \delta_{w_1,w_2-2} \delta_{\ell_1,\ell_2+1}  \nonumber \\ & + C_{w_2-2,\ell_2+1}  \delta_{w_1,w_2+2} \delta_{\ell_1,\ell_2+1}~.\nonumber
\end{align}

The upper diagonal matrix elements of the Casimir operator \(\hat{\overline{W}}^2 = \hat R_{+}\hat R_{-} + \hat \ell^2\) can then be expressed as
\begin{align}
\langle [N];w_2 \ell|\hat{\overline{W}}^2|[N];w_1 \ell\rangle &=
 A^2_{w_1,\ell_1} + B^2_{w_1,\ell_1} + C^2_{w_1,\ell_1} \, \delta_{w_2,w_1} \nonumber\\
+& (A_{w_1,\ell}B_{w_1+2,\ell} + C_{w_1,\ell}A_{w_1+2,\ell})\delta_{w_2,w_1+2}\label{Wbar2matel_so3}\\
+& C_{w_1,\ell}B_{w_1+4,\ell})\delta_{w_2,w_1+4}\nonumber~,
\end{align}
\noindent and the lower diagonal matrix elements can be computed from this equation.

The \(\hat W^2 \hat{\overline{W}}^2 + \hat{\overline{W}}^2\hat W^2 \) operator is then computed as in Eq. (\ref{n2W2matel_so3})

\begin{align}
\langle [N];w_2 \ell|\hat W^2 \hat{\overline{W}}^2 + \hat{\overline{W}}^2\hat W^2|[N];w_1 \ell\rangle &=
 2\omega_1(\omega_1+1) [\hat{\overline{W}}^2]_{w_1,w_1}\, \delta_{w_2,w_1} \nonumber\\
+& \left[\omega_1(\omega_1+1) + (\omega_1 +2)(\omega_1+3)\right][\hat {\overline{W}}^2]_{w_1,w_1+2}\delta_{w_2,w_1+2}\nonumber\\
+& \left[(\omega_1 - 2)(\omega_1-1) + \omega_1 (\omega_1+1)\right][\hat {\overline{W}}^2]_{w_1,w_1-2}\delta_{w_2,w_1-2}\label{W2W2barmatel_so3}\\
+& \left[\omega_1(\omega_1+1) + (\omega_1 +4)(\omega_1+5)\right][\hat {\overline{W}}^2]_{w_1,w_1+4}\delta_{w_2,w_1+4}\nonumber\\
+& \left[(\omega_1 - 4)(\omega_1-3) + \omega_1 (\omega_1+1)\right][\hat {\overline{W}}^2]_{w_1,w_1-4}\delta_{w_2,w_1-4}\nonumber~,
\end{align}
\noindent where \([\hat n^2]_{w_i,w_j}\) are the \(\hat n^2\) operator matrix elements.


\section{Programs included in the package}
\label{sec-3}

We proceed to describe the different programs provided. Compiled
\verb~Fortran~ programs have names with the suffix  \verb~_gfortran~ or \verb~_ifort~
depending on the compiler used.

 The
subroutine used for matrix diagonalization is \verb~LA_SYEVR~ from the \verb~Lapack95~ \cite{LAPACK95}, the \verb~Fortran 95~
interface to \verb~Lapack~ \cite{laug}. 

\subsection{\texttt{Fortran} programs to compute eigenvectors, eigenvalues and Inverse Participation Ratio (IPR)}
\label{sec-3-1}

The names of the programs \texttt{vibron2D\_[Ham]\_[Chain]\_[CompOpt]\_[Compiler]} mean:
\begin{itemize}
\item\relax [Ham]: Hamiltonian to be used. It must be replace by \texttt{Hmod}, \texttt{H2b} or \texttt{H4b} to choose one of the Hamiltonians of eqs. (\ref{modham1}), (\ref{ham2b}) or (\ref{H4b}) respectively.
\item\relax [Chain]: It must be \texttt{U3\_U2} (\texttt{U2\_SO3}) to use the linear (displaced oscillator) basis.
\item\relax [CompOpt]: This option is only available if the program has been compiled using the GNU compiler and allowing parallelization with openmp. In this case \texttt{[CompOpt]=openmp}.
\item\relax [Compiler]: Name of the compiler used. The source code has been written to  compile with GNU compiler (gfortran) or the intel's one (ifort).
\end{itemize}


For example, if the 2body Hamiltinian of 2DVM  is requeried, it's just necessary to type \texttt{vibron2D\_H2b\_U3\_U2\_gfortran} in order to use the Hamiltonian of eq. (\ref{ham2b}) diagonalized in the basis of Chain I and compiled sequentially.

Each program needs its own input file, in spite of sharing global variables. They are stored in \verb~test~ folder and have been based on \texttt{fortran} namelist. The lines \texttt{\&par\_aux} and \texttt{\&par\_0} are the same in the three files. The rest are related with the parameters of the chosen Hamiltonian.

\begin{verbatim}
#
#       INPUT FILE FOR 2DVM U(3) Calculations
#       Model Hamiltonian
#
&par_aux Iprint = 0, Eigenvec_Log = .T., Excitation_Log = .T., Save_avec_Log = .T.  /
&par_0   N_val = 10, L_min = 0, L_max = 5 /
&par_1   epsilon = 1.0,   xi = 0.4 /
\end{verbatim}

The line called \texttt{\&par\_aux} contains controls variables in order to choose the desired output:

\begin{itemize}
\item \verb~Iprint~ \emph{Integer variable}. Control of verbosity.
\item \verb~Eigenvec_Log~ \emph{Logical variable}.  If \verb~.True.~, The Inverse Participation Ratio (IPR) will be computed.
\item \verb~Excitation_Log~ \emph{Logical variable}. If \verb~.True.~, the excitation energies will be referred to the ground state.
\item \verb~Save_avec_Log~ \emph{Logical variable}. If \verb~.True.~, eigenstates will be saved in the current folder.
\end{itemize}


In the next line, known as \texttt{\&par\_0}, is where user can define the number of bosons \(N\) and the interval of angular momentum requiered:

\begin{itemize}
\item \verb~N_val~ \emph{Integer variable}. Totally symmetric irrep of \( U(3) \). Determines the Hilbert space size.
\item \verb~L_min~ and \verb~L_max~ \emph{Integer variables}. Vibrational angular momentum range \(\ell\in[\ell_{min},\ell_{max}]\).
\end{itemize}

As an example, we present the results of \texttt{vibron2D\_Hmod\_U3\_U2\_gfortran} setting \verb~Iprint~ to 3.

\begin{verbatim}
$ [...]/bin/vibron2D_Hmod_U3_U2_gfortran  /*Press enter*/
$ u3_Hmod.inp                             /*Type the name of the input file*/
\end{verbatim}

The screen will show:

\begin{verbatim}
Iprint =  3; Eigenvec_LOG =  T; Excitation_Log =  T; Save_Avec_Log =  T
N_val =     10; L_min =      0; L_max =      5
epsilon =  1.0000000E+00; xi =  4.0000000E-01
Operator number            1
Operator number            2
Operator number            1
Operator number            2
Operator number            1
Operator number            2
Operator number            1
Operator number            2
Operator number            1
Operator number            2
Operator number            1
Operator number            2
Operator number            1
Operator number            2
\end{verbatim}

The above information is related with the computing process and the input variables. The physical data are stored in the files called \texttt{autoval\_[...].dat} and \texttt{eigvec\_[...]}, where the name's tail is generated using the input information. For example: \texttt{autoval\_mh\_u2\_N10\_L0.dat} or \texttt{eigvec\_u2\_N10\_L0.dat}. With \verb~Iprint~\(>1\), both type of files displays the eigenvectors' coefficients. The output of \texttt{autoval\_mh\_u2\_N10\_L0.dat} with \verb~Iprint~\(=3\) is:

\begin{verbatim}
 Hamiltonian Matrix
  4.0000000000000000       0.84327404271156792        0.0000000000000000       ...      
 0.84327404271156792        3.8666666666666671        1.3303670708529569       ...     
  0.0000000000000000        1.3303670708529569        4.4444444444444446       ...     
  0.0000000000000000        0.0000000000000000        1.4605934866804429       ...      
  0.0000000000000000        0.0000000000000000        0.0000000000000000       ...     
  0.0000000000000000        0.0000000000000000        0.0000000000000000       ...      
GS_energy =    2.2741144458696825     
L_val =            0
          0   4.4408920985006262E-016   2.8418774855178928     
-0.32933743241225288      |           0 >
 0.67403796185517229      |           2 >
-0.59811984409478036      |           4 >
 0.27481947241670696      |           6 >
 -6.2557421702539709E-002 |           8 >
  4.8125108622299488E-003 |          10 >
          2   1.4773749338491529        2.6037467907764698     
 0.74575014268070683      |           0 >
-0.21977058600834720      |           2 >
-0.45367866587132649      |           4 >
 0.41541639448225348      |           6 >
-0.13043184131934032      |           8 >
  1.2248931213675825E-002 |          10 >
          4   2.6183838541904825        3.3349012803683000     

  ...
\end{verbatim}

The first lines (truncated) show the terms of the Hamiltonian matrix. After that, the program displays information about the Zero Point Energy (ZPE) and the angular momentum. The last part alternates a line that contains the vibrational (in this case) quantum number, the energy and the IPR of the levels with the eigenstate in the choosen basis. A simpler version of this output could be obtained setting \verb~Iprint~ to \(0\) (\textbf{recommended}):

\begin{verbatim}
 0   4.4408920985006262E-016   2.8418774855178928     
 2   1.4773749338491529        2.6037467907764698     
 4   2.6183838541904825        3.3349012803683000     
 6   4.1112373940234130        3.7060015095032330     
 8   6.0440884180968331        2.0756877528461386     
10   8.3264509468442469        1.1333042664402908
\end{verbatim}

The \texttt{eigvec\_u2\_N10\_L0.dat} file with \verb~Iprint~\(=3\) contains similar information but different structure. This file will not be generate if \verb~Save_avec_Log~ is false.

\begin{verbatim}
# N =           10 ; L =            0  ; u2 basis
# Eigenvalues 
  4.4408920985006262E-016   1.4773749338491529        2.6183838541904825       ...  
# Hamiltonian Diagonal 
  1.7258855541303175        1.5925522207969847        2.1703299985747622       ... 
-0.32933743241225288       0.74575014268070683       0.56404597555589175       ... 
 0.67403796185517229      -0.21977058600834720       0.59697091199520735       ... 
-0.59811984409478036      -0.45367866587132649       0.10278840979622948       ... 
 0.27481947241670696       0.41541639448225348      -0.51221349887982404       ... 
 -6.2557421702539709E-002 -0.13043184131934032       0.22778225002450472       ...    
  4.8125108622299488E-003   1.2248931213675825E-002  -2.5787373682437546E-002  ...
\end{verbatim}

The first line is the information about the system. Next two consist of the eigenvalues and Hamiltonian diagonal. The rest lines are the n-th eigenstate coefficients refer to the chosen basis.





\subsection{Programs to fit experimental vibrational bending spectra}
\label{sec-3-2}
This part of the software has two different purposes depending on the input definitions, so, before explaining them, we present experimental energies and input files.
\subsubsection{Program \texttt{chi2\_U3}}
\label{sec-3-2-1}
\label{chi2p}
    \verb~2D Vibron Model Program Suite~ includes a \verb~chi2_U3~ program to compute the \(\chi^2\) and use it in different minimization methods.

This part of the software has two different purposes depending on the input definitions, so, before explaining them, we present experimental energies and input files.

First of all we need to collect the experimental data into a file. In this example we use the \verb~hydrogen isocyanide~ (HNC) molecule \cite{MellauHNC,BunkerHNC}.

\begin{verbatim}
19
462.72    1.0   1    1  
926.50    1.0   2    0  
936.05    1.0   2    2  
1398.56   1.0   3    1 
1419.97   1.0   3    3 
1867.05   1.0   4    0 
1878.72   1.0   4    2 
1913.87   1.0   4    4 
2341.84   1.0   5    1 
2366.83   1.0   5    3 
2417.57   1.0   5    5 
2809.29   1.0   6    0 
2822.75   1.0   6    2 
2863.11   1.0   6    4 
2930.90   1.0   6    6 
3281.50   1.0   7    1 
3309.78   1.0   7    3 
3367.37   1.0   7    5 
3453.78   1.0   7    7
\end{verbatim}

The first line must be the total number of experimental data. After that, the file is structured in 4 columns. The first column corresponds with the energy, the next one to the experimental error of data and, the last two are the quantum numbers. If you want to include any data in the prediction but not at the fitting time, you can set the experiemntal error to 0.0 . The energy's label must be in accordance with the nature of the molecule: linear, \(U(2)\) basis \(\ket{n\ell}\), or bent, \(SO(3)\) \(\ket{{\nu_b}^{\ell}}\).

Once finished, the next step is to construct the input file:

\begin{verbatim}
 &INP0 BENT=.F., exp_data_file='exp_HNC_Danielle.dat' / 
 &INP1 N_val=40, LMAX=7, VMAX=7, EMINL=.F. /
#
# PARAMETERS THAT TUNE THE OUTPUT
#
 &INP2 IPRINT=0, DIS_RES = *** /
 &INP1b P11=1414.0D0 /
 &INP2b P21=-29.837D0, P22=15.81D0, P23=-8.054D0 /
 &INP3b P31=0.0D0, P32=4.9D-2, P33=0.0D0 /
 &INP4b P41=0.0D0, P42=0.0D0, P43=0.0D0, P44=0.0D0, P45=0.0D0, P46=0.0D0, P47=0.0D0 /
\end{verbatim}

\begin{itemize}
\item \verb~BENT~: \verb~.T.~ (\verb~.F.~) if the molecule is bent (linear).
\item \verb~exp_data_file~: The name of the experimental energies' file.
\item \verb~N_val~: Vibron number. Determines the Hilbert space size.
\item \verb~LMAX~: The highest angular momentum number \(\ell\) in \verb~exp_data_file~.
\item \verb~VMAX~: The highest harmonic number \(n\) for the linear case or the bending number \(\nu_b\) when \verb~BENT~ is true.
\item \verb~EMINL~: Fix the lowest level at the zero energy.
\item \verb~Iprint~: Integer variable. Controls the program's verbosity.
\item \verb~DIS_RES~: Match as true (.T.) to display residuals energies or false (.F.) to get only the \(\chi^2\).
\item \(P_{ij}\): Initial value of the parameter \(P_{ij}\)

Running the program with \verb~DIS_RES=.T.~:
\end{itemize}

\begin{verbatim}
$ echo fit.inp | ../../bin/chi2_U3_gfortran 


   7.0788432691415437E-003
  -2.7776246656685544E-004
   9.1877661752732820E-003
  -3.3695279357061736E-002
  -3.0352671044056478E-002
  -8.4165727005256485E-002
  -4.9249332600084017E-002
   5.6619175268906474E-002
  -3.3439091556601852E-002
  -4.1176236554747447E-002
  -5.0164535727162729E-002
   7.7310508308073622E-002
  0.16723874181343490     
  -2.9727267470207153E-002
   1.0623963672060199E-002
   5.5108955660216452E-002
 -0.11479814853555581     
  0.16486361585430132     
  0.19601871556506012
\end{verbatim}

On the other hand, if you set \verb~DIS_RES~ to false:

\begin{verbatim}
$ echo fit.inp | ../../bin/chi2_U3_gfortran 

  0.13699774310846663
\end{verbatim}


\subsubsection{Minimization using the \texttt{Python-script}}
\label{sec-3-2-2}

\subsubsection{Minimization using \texttt{Minuit-CERN} package}
\label{sec-3-2-3}

In this section we are going to explain how to use this minimization method, where \verb~Minuit-CERN~ package \cite{minuit} has been used. Most of the inputs items have been detailed at subsection \ref{chi2p}. The energies file have the same structure and the input file is similar changing the meaning of \verb~DIS_RES~:

\begin{verbatim}
 &INP0 BENT=.F., exp_data_file='exp_HNC.dat', output_file_0="output_HNC" / 
 &INP1 N_val=40, LMAX=7, VMAX=7, EMINL=.F. /
#
# PARAMETERS THAT TUNE THE OUTPUT
#
 &INP2 IPRINT=0, DIS_RES = .F. /
 &INP1b P11=740.0D0 /
 &INP2b P21=-3.597D0, P22=1.5D0, P23=-7.2D0 /
 &INP3b P31=0.0D0, P32=0.002D0, P33=0.0D0 /
 &INP4b P41=0.0D0, P42=0.0D0, P43=0.0D0, P44=0.0D0, P45=0.0D0, P46=0.0D0, P47=0.0D0 /
 &fix_par fixed_par = 'fix  5 7 8 9 10 11 12 13 14' /
 
 # P11 ---> 1
 # P21 ---> 2
 # P22 ---> 3
 # P23 ---> 4
 # P31 ---> 5
 # P32 ---> 6
 # P33 ---> 7
 # P41 ---> 8
 # P42 ---> 9
 # P43 ---> 10
 # P44 ---> 11
 # P45 ---> 12
 # P46 ---> 13
 # P47 ---> 14
\end{verbatim}

Most of definitions have been explained in previous sections, the others mean:

\begin{itemize}
\item \verb~BENT~: \verb~.T.~ (\verb~.F.~) if the molecule is bent (linear).
\item \verb~exp_data_file~: The name of the experimental energies' file.
\item \verb~output_file_0~: To choose the name of the file where the fit report will be saved.
\item \verb~LMAX~: The highest angular momentum number \(\ell\) in \verb~exp_data_file~.
\item \verb~VMAX~: The highest harmonic number \(n\) for the linear case or the bending number \(\nu_b\) when \verb~BENT~ is true.
\item \verb~DIS_RES~: Match as true (.T.) to display residuals energies.
\item \(P_{ij}\): Initial value of the parameter \(P_{ij}\)
\item \verb~fixed_par~: \verb~Minuit~ order. This string of characters consists of different integers that are associated with the \(P_{ij}\) parameters. If one of them appears, \verb~Minuit~ will keep it fixed.
\end{itemize}

If the program \verb~Min_minuit_gfortran~ is executed as mentioned:

\begin{verbatim}
$ echo fit.inp | ../bin/Min_minuit_gfortran

Note: The following floating-point exceptions are signalling: IEEE_INVALID_FLAG IEEE_DIVIDE_BY_ZERO IEEE_UNDERFLOW_FLAG IEEE_DENORMAL

$ tail -n 43 output_HNC

  EXT PARAMETER                APPROXIMATE        STEP         FIRST   
  NO.   NAME        VALUE          ERROR          SIZE      DERIVATIVE 
   1     P11        1413.8       0.37063       0.67417E-03  -0.14902    
   2     P21       -29.832       0.15233E-01   0.14225E-04   -1.8667    
   3     P22        15.807       0.10429       0.75373E-05  -0.14985    
   4     P23       -8.0528       0.28648E-02   0.38399E-05   -13.647    
   5     P31        0.0000         fixed    
   6     P32       0.48757E-01   0.95612E-02   0.65872E-06   -1.3627    
   7     P33        0.0000         fixed    
   8     P41        0.0000         fixed    
   9     P42        0.0000         fixed    
  10     P43        0.0000         fixed    
  11     P44        0.0000         fixed    
  12     P45        0.0000         fixed    
  13     P46        0.0000         fixed    
  14     P47        0.0000         fixed    

 EXTERNAL ERROR MATRIX.    NDIM=  50    NPAR=  5    ERR DEF=  1.00    
  0.137E+00-0.432E-02-0.926E-03-0.915E-03 0.200E-03
 -0.432E-02 0.232E-03 0.563E-03 0.154E-04-0.537E-04
 -0.926E-03 0.563E-03 0.109E-01-0.862E-04-0.978E-03
 -0.915E-03 0.154E-04-0.862E-04 0.821E-05 0.628E-05
  0.200E-03-0.537E-04-0.978E-03 0.628E-05 0.914E-04
 ERR MATRIX NOT POS-DEF

 PARAMETER  CORRELATION COEFFICIENTS
       NO.  GLOBAL     1     2     3     4     6
        1  0.99666  1.000-0.766-0.024-0.861 0.056
        2  0.98707 -0.766 1.000 0.355 0.353-0.369
        3  0.99201 -0.024 0.355 1.000-0.288-0.980
        4  0.99374 -0.861 0.353-0.288 1.000 0.229
        6  0.99086  0.056-0.369-0.980 0.229 1.000
 ERR MATRIX NOT POS-DEF
minuit>  **********
 **    4 **CALL    3.000    
 **********
  ___________________________________ 
 | rms =    7.9576384169698269E-002  |
 |___________________________________|
minuit>  **********
 **    5 **EXIT 
 **********
 ..........MINUIT TERMINATED BY MINUIT COMMAND: EXIT 

$
\end{verbatim}

\subsection{\texttt{Perl} scripts}
\label{sec-3-3}



\printbibliography
\end{document}
